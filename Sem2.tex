\documentclass[11,]{article}
\usepackage{lmodern}
\usepackage{amssymb,amsmath}
\usepackage{ifxetex,ifluatex}
\usepackage{fixltx2e} % provides \textsubscript
\ifnum 0\ifxetex 1\fi\ifluatex 1\fi=0 % if pdftex
  \usepackage[T1]{fontenc}
  \usepackage[utf8]{inputenc}
\else % if luatex or xelatex
  \ifxetex
    \usepackage{mathspec}
  \else
    \usepackage{fontspec}
  \fi
  \defaultfontfeatures{Ligatures=TeX,Scale=MatchLowercase}
  \newcommand{\euro}{€}
\fi
% use upquote if available, for straight quotes in verbatim environments
\IfFileExists{upquote.sty}{\usepackage{upquote}}{}
% use microtype if available
\IfFileExists{microtype.sty}{%
\usepackage{microtype}
\UseMicrotypeSet[protrusion]{basicmath} % disable protrusion for tt fonts
}{}
\usepackage[margin=1in]{geometry}
\usepackage{hyperref}
\PassOptionsToPackage{usenames,dvipsnames}{color} % color is loaded by hyperref
\hypersetup{unicode=true,
            pdftitle={CompStat/R - Paper 2},
            pdfauthor={Group 2: Carlo Michaelis, Patrick Molligo, Lukas Ruff},
            pdfborder={0 0 0},
            breaklinks=true}
\urlstyle{same}  % don't use monospace font for urls
\usepackage{color}
\usepackage{fancyvrb}
\newcommand{\VerbBar}{|}
\newcommand{\VERB}{\Verb[commandchars=\\\{\}]}
\DefineVerbatimEnvironment{Highlighting}{Verbatim}{commandchars=\\\{\}}
% Add ',fontsize=\small' for more characters per line
\usepackage{framed}
\definecolor{shadecolor}{RGB}{248,248,248}
\newenvironment{Shaded}{\begin{snugshade}}{\end{snugshade}}
\newcommand{\KeywordTok}[1]{\textcolor[rgb]{0.13,0.29,0.53}{\textbf{{#1}}}}
\newcommand{\DataTypeTok}[1]{\textcolor[rgb]{0.13,0.29,0.53}{{#1}}}
\newcommand{\DecValTok}[1]{\textcolor[rgb]{0.00,0.00,0.81}{{#1}}}
\newcommand{\BaseNTok}[1]{\textcolor[rgb]{0.00,0.00,0.81}{{#1}}}
\newcommand{\FloatTok}[1]{\textcolor[rgb]{0.00,0.00,0.81}{{#1}}}
\newcommand{\ConstantTok}[1]{\textcolor[rgb]{0.00,0.00,0.00}{{#1}}}
\newcommand{\CharTok}[1]{\textcolor[rgb]{0.31,0.60,0.02}{{#1}}}
\newcommand{\SpecialCharTok}[1]{\textcolor[rgb]{0.00,0.00,0.00}{{#1}}}
\newcommand{\StringTok}[1]{\textcolor[rgb]{0.31,0.60,0.02}{{#1}}}
\newcommand{\VerbatimStringTok}[1]{\textcolor[rgb]{0.31,0.60,0.02}{{#1}}}
\newcommand{\SpecialStringTok}[1]{\textcolor[rgb]{0.31,0.60,0.02}{{#1}}}
\newcommand{\ImportTok}[1]{{#1}}
\newcommand{\CommentTok}[1]{\textcolor[rgb]{0.56,0.35,0.01}{\textit{{#1}}}}
\newcommand{\DocumentationTok}[1]{\textcolor[rgb]{0.56,0.35,0.01}{\textbf{\textit{{#1}}}}}
\newcommand{\AnnotationTok}[1]{\textcolor[rgb]{0.56,0.35,0.01}{\textbf{\textit{{#1}}}}}
\newcommand{\CommentVarTok}[1]{\textcolor[rgb]{0.56,0.35,0.01}{\textbf{\textit{{#1}}}}}
\newcommand{\OtherTok}[1]{\textcolor[rgb]{0.56,0.35,0.01}{{#1}}}
\newcommand{\FunctionTok}[1]{\textcolor[rgb]{0.00,0.00,0.00}{{#1}}}
\newcommand{\VariableTok}[1]{\textcolor[rgb]{0.00,0.00,0.00}{{#1}}}
\newcommand{\ControlFlowTok}[1]{\textcolor[rgb]{0.13,0.29,0.53}{\textbf{{#1}}}}
\newcommand{\OperatorTok}[1]{\textcolor[rgb]{0.81,0.36,0.00}{\textbf{{#1}}}}
\newcommand{\BuiltInTok}[1]{{#1}}
\newcommand{\ExtensionTok}[1]{{#1}}
\newcommand{\PreprocessorTok}[1]{\textcolor[rgb]{0.56,0.35,0.01}{\textit{{#1}}}}
\newcommand{\AttributeTok}[1]{\textcolor[rgb]{0.77,0.63,0.00}{{#1}}}
\newcommand{\RegionMarkerTok}[1]{{#1}}
\newcommand{\InformationTok}[1]{\textcolor[rgb]{0.56,0.35,0.01}{\textbf{\textit{{#1}}}}}
\newcommand{\WarningTok}[1]{\textcolor[rgb]{0.56,0.35,0.01}{\textbf{\textit{{#1}}}}}
\newcommand{\AlertTok}[1]{\textcolor[rgb]{0.94,0.16,0.16}{{#1}}}
\newcommand{\ErrorTok}[1]{\textcolor[rgb]{0.64,0.00,0.00}{\textbf{{#1}}}}
\newcommand{\NormalTok}[1]{{#1}}
\usepackage{graphicx,grffile}
\makeatletter
\def\maxwidth{\ifdim\Gin@nat@width>\linewidth\linewidth\else\Gin@nat@width\fi}
\def\maxheight{\ifdim\Gin@nat@height>\textheight\textheight\else\Gin@nat@height\fi}
\makeatother
% Scale images if necessary, so that they will not overflow the page
% margins by default, and it is still possible to overwrite the defaults
% using explicit options in \includegraphics[width, height, ...]{}
\setkeys{Gin}{width=\maxwidth,height=\maxheight,keepaspectratio}
\setlength{\parindent}{0pt}
\setlength{\parskip}{6pt plus 2pt minus 1pt}
\setlength{\emergencystretch}{3em}  % prevent overfull lines
\providecommand{\tightlist}{%
  \setlength{\itemsep}{0pt}\setlength{\parskip}{0pt}}
\setcounter{secnumdepth}{0}

%%% Use protect on footnotes to avoid problems with footnotes in titles
\let\rmarkdownfootnote\footnote%
\def\footnote{\protect\rmarkdownfootnote}

%%% Change title format to be more compact
\usepackage{titling}

% Create subtitle command for use in maketitle
\newcommand{\subtitle}[1]{
  \posttitle{
    \begin{center}\large#1\end{center}
    }
}

\setlength{\droptitle}{-2em}
  \title{CompStat/R - Paper 2}
  \pretitle{\vspace{\droptitle}\centering\huge}
  \posttitle{\par}
  \author{Group 2: Carlo Michaelis, Patrick Molligo, Lukas Ruff}
  \preauthor{\centering\large\emph}
  \postauthor{\par}
  \predate{\centering\large\emph}
  \postdate{\par}
  \date{21 June 2016}



% Redefines (sub)paragraphs to behave more like sections
\ifx\paragraph\undefined\else
\let\oldparagraph\paragraph
\renewcommand{\paragraph}[1]{\oldparagraph{#1}\mbox{}}
\fi
\ifx\subparagraph\undefined\else
\let\oldsubparagraph\subparagraph
\renewcommand{\subparagraph}[1]{\oldsubparagraph{#1}\mbox{}}
\fi

\begin{document}
\maketitle

\subsection{Part I: Functions}\label{part-i-functions}

\subsubsection{Functions I}\label{functions-i}

Below we define a function \texttt{dropNa} which, given an atomic vector
\texttt{x} as an argument, returns \texttt{x} after removing missing
values.

\begin{Shaded}
\begin{Highlighting}[]
\NormalTok{dropNa <-}\StringTok{ }\NormalTok{function(x) \{}
  \CommentTok{# Expects an atomic vector as an argument and returns it without missing}
  \CommentTok{# values}
  \CommentTok{# }
  \CommentTok{# Args:}
  \CommentTok{#   x: atomic vector}
  \CommentTok{#}
  \CommentTok{# Returns:}
  \CommentTok{#   The atomic vector x without missing values}
  
  \CommentTok{# To remove the NAs, we use simple logical subsetting}
  \NormalTok{y <-}\StringTok{ }\NormalTok{x[!}\KeywordTok{is.na}\NormalTok{(x)]}
  
  \CommentTok{# Return y}
  \NormalTok{y}
\NormalTok{\}}
\end{Highlighting}
\end{Shaded}

Let's test our implementation with the following line of code:

\begin{Shaded}
\begin{Highlighting}[]
\KeywordTok{all.equal}\NormalTok{(}\KeywordTok{dropNa}\NormalTok{(}\KeywordTok{c}\NormalTok{(}\DecValTok{1}\NormalTok{, }\DecValTok{2}\NormalTok{, }\DecValTok{3}\NormalTok{, }\OtherTok{NA}\NormalTok{, }\DecValTok{1}\NormalTok{, }\DecValTok{2}\NormalTok{, }\DecValTok{3}\NormalTok{)), }\KeywordTok{c}\NormalTok{(}\DecValTok{1}\NormalTok{, }\DecValTok{2}\NormalTok{, }\DecValTok{3}\NormalTok{, }\DecValTok{1}\NormalTok{, }\DecValTok{2}\NormalTok{, }\DecValTok{3}\NormalTok{))}
\end{Highlighting}
\end{Shaded}

\begin{verbatim}
## [1] TRUE
\end{verbatim}

As we can see from this positive test, our implementation was
successful.

\subsubsection{Functions II}\label{functions-ii}

\paragraph{Part I}\label{part-i}

Below we define a function \texttt{meanVarSdSe} which, given a numeric
vector \texttt{x} as an argument, returns the mean, the variance, the
standard deviation and the standard error of \texttt{x}.

\begin{Shaded}
\begin{Highlighting}[]
\NormalTok{meanVarSdSe <-}\StringTok{ }\NormalTok{function(x) \{}
  \CommentTok{# Expects a numeric vector as an argument and returns the mean,}
  \CommentTok{# the variance, the standard deviation and the standard error}
  \CommentTok{# }
  \CommentTok{# Args:}
  \CommentTok{#   x: numeric vector}
  \CommentTok{#}
  \CommentTok{# Returns:}
  \CommentTok{#   a numerical vector containing mean, variance, standard deviation}
  \CommentTok{#   and standard error of x}
  
  \CommentTok{# We check if x is numeric vector}
  \CommentTok{# If not: stop and throw error}
  \NormalTok{if( !}\KeywordTok{is.numeric}\NormalTok{(x) ) \{}
    \KeywordTok{stop}\NormalTok{(}\StringTok{"Argument needs to be numeric."}\NormalTok{)}
  \NormalTok{\}}
  
  \CommentTok{# Create vector object}
  \NormalTok{y <-}\StringTok{ }\KeywordTok{vector}\NormalTok{()}
  
  \CommentTok{# Calculate mean, variance, standard deviation and standard error}
  \CommentTok{# and save it in y}
  \NormalTok{y[}\DecValTok{1}\NormalTok{] <-}\StringTok{ }\KeywordTok{mean}\NormalTok{(x)}
  \NormalTok{y[}\DecValTok{2}\NormalTok{] <-}\StringTok{ }\KeywordTok{var}\NormalTok{(x)}
  \NormalTok{y[}\DecValTok{3}\NormalTok{] <-}\StringTok{ }\KeywordTok{sd}\NormalTok{(x)}
  \NormalTok{y[}\DecValTok{4}\NormalTok{] <-}\StringTok{ }\NormalTok{y[}\DecValTok{3}\NormalTok{]/}\KeywordTok{sqrt}\NormalTok{(}\KeywordTok{length}\NormalTok{(x))}
  
  \CommentTok{# Set names to vector entries}
  \KeywordTok{names}\NormalTok{(y) <-}\StringTok{ }\KeywordTok{c}\NormalTok{(}\StringTok{"mean"}\NormalTok{, }\StringTok{"var"}\NormalTok{, }\StringTok{"sd"}\NormalTok{, }\StringTok{"se"}\NormalTok{)}
  
  \CommentTok{# Return the numeric vector y}
  \NormalTok{y}
\NormalTok{\}}
\end{Highlighting}
\end{Shaded}

To test the function, we define a numeric vector, which contains numbers
from \(1\) to \(100\), and use it as an argument for our function
\texttt{meanVarSdSe}:

\begin{Shaded}
\begin{Highlighting}[]
\NormalTok{x <-}\StringTok{ }\DecValTok{1}\NormalTok{:}\DecValTok{100}
\KeywordTok{meanVarSdSe}\NormalTok{(x)}
\end{Highlighting}
\end{Shaded}

\begin{verbatim}
##       mean        var         sd         se 
##  50.500000 841.666667  29.011492   2.901149
\end{verbatim}

Finally we can confirm that the result is of class \texttt{numeric}:

\begin{Shaded}
\begin{Highlighting}[]
\KeywordTok{class}\NormalTok{(}\KeywordTok{meanVarSdSe}\NormalTok{(x))}
\end{Highlighting}
\end{Shaded}

\begin{verbatim}
## [1] "numeric"
\end{verbatim}

\paragraph{Part II}\label{part-ii}

Now we will have a look at the case below. We would expect that the
function will return a vector with \texttt{NA}s:

\begin{Shaded}
\begin{Highlighting}[]
\NormalTok{x <-}\StringTok{ }\KeywordTok{c}\NormalTok{(}\OtherTok{NA}\NormalTok{, }\DecValTok{1}\NormalTok{:}\DecValTok{100}\NormalTok{)}
\KeywordTok{meanVarSdSe}\NormalTok{(x)}
\end{Highlighting}
\end{Shaded}

\begin{verbatim}
## mean  var   sd   se 
##   NA   NA   NA   NA
\end{verbatim}

The reason for the result is that the functions \texttt{mean()},
\texttt{var()} and \texttt{sd()} use \texttt{na.rm\ =\ FALSE} as
default, which means that missing values are not removed. If the vector
\texttt{x} contains a missing value, the \texttt{mean()} function (as
well as \texttt{var()} and \texttt{sd()}) will just return \texttt{NA}
to inform about missing values. In the case of calculating standard
error we use the result from our \texttt{sd()} function and calculate an
\texttt{NA} value with some other numeric values, which will ultimately
result in \texttt{NA} again.

To solve the problem, we can add \texttt{na.rm\ =\ TRUE} to these three
functions. To make this optional, we will improve the
\texttt{meanVarSdSe} function from above as follows:

\begin{Shaded}
\begin{Highlighting}[]
\NormalTok{meanVarSdSe <-}\StringTok{ }\NormalTok{function(x, ...) \{}
  \CommentTok{# Expects a numeric vector and flag to handle missing values as an argument}
  \CommentTok{# and returns the mean, the variance, the standard deviation}
  \CommentTok{# and the standard error}
  \CommentTok{# }
  \CommentTok{# Args:}
  \CommentTok{#   x: numeric vector, na.rm: boolean}
  \CommentTok{#}
  \CommentTok{# Returns:}
  \CommentTok{#   a numerical vector containing mean, variance, standard deviation}
  \CommentTok{#   and standard error of x}
  
  \CommentTok{# We check if x is numeric vector}
  \CommentTok{# If not: stop and throw error}
  \NormalTok{if( !}\KeywordTok{is.numeric}\NormalTok{(x) ) \{}
    \KeywordTok{stop}\NormalTok{(}\StringTok{"Argument needs to be numeric."}\NormalTok{)}
  \NormalTok{\}}
  
  \CommentTok{# Create vector object}
  \NormalTok{y <-}\StringTok{ }\KeywordTok{vector}\NormalTok{()}
  
  \CommentTok{# Calculate mean, variance, standard deviation and standard error}
  \CommentTok{# and save it in y}
  \NormalTok{y[}\DecValTok{1}\NormalTok{] <-}\StringTok{ }\KeywordTok{mean}\NormalTok{(x, ...)}
  \NormalTok{y[}\DecValTok{2}\NormalTok{] <-}\StringTok{ }\KeywordTok{var}\NormalTok{(x, ...)}
  \NormalTok{y[}\DecValTok{3}\NormalTok{] <-}\StringTok{ }\KeywordTok{sd}\NormalTok{(x, ...)}
  \NormalTok{y[}\DecValTok{4}\NormalTok{] <-}\StringTok{ }\NormalTok{y[}\DecValTok{3}\NormalTok{]/}\KeywordTok{sqrt}\NormalTok{(}\KeywordTok{length}\NormalTok{(x) -}\StringTok{ }\KeywordTok{sum}\NormalTok{(}\KeywordTok{is.na}\NormalTok{(x)))}
  
  \CommentTok{# Set names to vector entries}
  \KeywordTok{names}\NormalTok{(y) <-}\StringTok{ }\KeywordTok{c}\NormalTok{(}\StringTok{"mean"}\NormalTok{, }\StringTok{"var"}\NormalTok{, }\StringTok{"sd"}\NormalTok{, }\StringTok{"se"}\NormalTok{)}
  
  \CommentTok{# Return the numeric vector y}
  \NormalTok{y}
\NormalTok{\}}
\end{Highlighting}
\end{Shaded}

We define the function with an ellipse \texttt{...}. Our function can
now receive multiple arguments after the first input \texttt{x}. These
arguments are used in \texttt{mean()}, \texttt{var()} and \texttt{sd()}.
If we want to remove missing values in all of these functions (to get a
result in the case of missing values), we can pass
\texttt{na.rm\ =\ TRUE} as another argument, such as here:
\texttt{meanVarSdSe(x,\ na.rm\ =\ TRUE)}. We just have to be aware of
\texttt{length(x)} in this case. If we want to have the same result as
above we have to remove the sum of \texttt{NA} values from the length of
\texttt{x}. Otherwise the function will calculate a different result
than in Part I, because then lentgh differs.

Let's confirm the result:

\begin{Shaded}
\begin{Highlighting}[]
\KeywordTok{meanVarSdSe}\NormalTok{(}\KeywordTok{c}\NormalTok{(x, }\OtherTok{NA}\NormalTok{), }\DataTypeTok{na.rm =} \OtherTok{TRUE}\NormalTok{)}
\end{Highlighting}
\end{Shaded}

\begin{verbatim}
##       mean        var         sd         se 
##  50.500000 841.666667  29.011492   2.901149
\end{verbatim}

\paragraph{Part III}\label{part-iii}

Now we will use the function \texttt{dropNa} from Functions I to deal
with missing values in \texttt{meanVarSdSe}.

\begin{Shaded}
\begin{Highlighting}[]
\NormalTok{meanVarSdSe <-}\StringTok{ }\NormalTok{function(x) \{}
  \CommentTok{# Expects a numeric vector as an argument and returns the mean,}
  \CommentTok{# the variance, the standard deviation and the standard error}
  \CommentTok{# it also removes missing values if x contains some}
  \CommentTok{# }
  \CommentTok{# Args:}
  \CommentTok{#   x: numeric vector}
  \CommentTok{#}
  \CommentTok{# Returns:}
  \CommentTok{#   a numerical vector containing mean, variance, standard deviation}
  \CommentTok{#   and standard error of x}
  
  \CommentTok{# We check if x is numeric vector}
  \CommentTok{# If not: stop and throw error}
  \NormalTok{if( !}\KeywordTok{is.numeric}\NormalTok{(x) ) \{}
    \KeywordTok{stop}\NormalTok{(}\StringTok{"Argument needs to be numeric."}\NormalTok{)}
  \NormalTok{\}}
  
  \CommentTok{# We check if x contains missing values}
  \CommentTok{# If so: remove missing values using dropNA}
  \NormalTok{if( }\KeywordTok{sum}\NormalTok{(}\KeywordTok{is.na}\NormalTok{(x)) >}\StringTok{ }\DecValTok{0} \NormalTok{) \{}
    \NormalTok{x <-}\StringTok{ }\KeywordTok{dropNa}\NormalTok{(x)}
  \NormalTok{\}}
  
  \CommentTok{# Create vector object}
  \NormalTok{y <-}\StringTok{ }\KeywordTok{vector}\NormalTok{()}
  
  \CommentTok{# Calculate mean, variance, standard deviation and standard error}
  \CommentTok{# and save it in y}
  \NormalTok{y[}\DecValTok{1}\NormalTok{] <-}\StringTok{ }\KeywordTok{mean}\NormalTok{(x)}
  \NormalTok{y[}\DecValTok{2}\NormalTok{] <-}\StringTok{ }\KeywordTok{var}\NormalTok{(x)}
  \NormalTok{y[}\DecValTok{3}\NormalTok{] <-}\StringTok{ }\KeywordTok{sd}\NormalTok{(x)}
  \NormalTok{y[}\DecValTok{4}\NormalTok{] <-}\StringTok{ }\NormalTok{y[}\DecValTok{3}\NormalTok{]/}\KeywordTok{sqrt}\NormalTok{(}\KeywordTok{length}\NormalTok{(x))}
  
  \CommentTok{# Set names to vector entries}
  \KeywordTok{names}\NormalTok{(y) <-}\StringTok{ }\KeywordTok{c}\NormalTok{(}\StringTok{"mean"}\NormalTok{, }\StringTok{"var"}\NormalTok{, }\StringTok{"sd"}\NormalTok{, }\StringTok{"se"}\NormalTok{)}
  
  \CommentTok{# Return the numeric vector y}
  \NormalTok{y}
\NormalTok{\}}
\end{Highlighting}
\end{Shaded}

We used the function from Part I and added a condition which checks if
we have missing values in \texttt{x}, using \texttt{is.na}. If the sum
of \texttt{NA} values is greater than \(0\) (i.e.,if there is one or
more missing value), we use the function \texttt{dropNA} from the first
exercise to remove all missing values. The remaining code of the
function can remain as above in Part I.

We can confirm the result:

\begin{Shaded}
\begin{Highlighting}[]
\KeywordTok{meanVarSdSe}\NormalTok{(}\KeywordTok{c}\NormalTok{(x, }\OtherTok{NA}\NormalTok{))}
\end{Highlighting}
\end{Shaded}

\begin{verbatim}
##       mean        var         sd         se 
##  50.500000 841.666667  29.011492   2.901149
\end{verbatim}

\subsubsection{Functions III}\label{functions-iii}

In this section we define an infix function \texttt{\%or\%}. This
function should behave like the logical operator \texttt{\textbar{}}.

\begin{Shaded}
\begin{Highlighting}[]
\CommentTok{# Define infix function %or%}
\StringTok{`}\DataTypeTok\StringTok{`} \NormalTok{<-}\StringTok{ }\NormalTok{function(a, b) \{}
  \CommentTok{# Check if vector a and b is logical}
  \NormalTok{if( !(}\KeywordTok{is.logical}\NormalTok{(a) &}\StringTok{ }\KeywordTok{is.logical}\NormalTok{(b)) ) \{}
    \KeywordTok{stop}\NormalTok{(}\StringTok{"a and/or b have to be logical vectors."}\NormalTok{)}
  \NormalTok{\}}
  
  \CommentTok{# Use ifelse to calculate result and return it directly}
  \CommentTok{# If the sum of the entry of vector a and the entry of vector b}
  \CommentTok{# is greater than or equal to 1, set result to TRUE, otherwise to FALSE}
  \KeywordTok{ifelse}\NormalTok{(a +}\StringTok{ }\NormalTok{b >=}\StringTok{ }\DecValTok{1}\NormalTok{,  }\OtherTok{TRUE}\NormalTok{, }\OtherTok{FALSE}\NormalTok{)}
\NormalTok{\}}
\end{Highlighting}
\end{Shaded}

First we check if we have logical vectors. If \texttt{a} and/or
\texttt{b} are not logical, we leave the function and throw an error.
Otherwise we can calculate the \texttt{or} operation using the
\texttt{ifelse} function and return the result directly after
calculation. Inside of the \texttt{ifelse} function, the first argument
checks the condition if the sum of the values \texttt{a} and \texttt{b}
are greater than or equal to \(1\), where \texttt{TRUE} is equal to
\(1\) and \texttt{FALSE} is equal to \(0\).

To confirm the function, we test an example:

\begin{Shaded}
\begin{Highlighting}[]
\KeywordTok{c}\NormalTok{(}\OtherTok{TRUE}\NormalTok{, }\OtherTok{FALSE}\NormalTok{, }\OtherTok{TRUE}\NormalTok{, }\OtherTok{FALSE}\NormalTok{) %or%}\StringTok{ }\KeywordTok{c}\NormalTok{(}\OtherTok{TRUE}\NormalTok{, }\OtherTok{TRUE}\NormalTok{, }\OtherTok{FALSE}\NormalTok{, }\OtherTok{FALSE}\NormalTok{)}
\end{Highlighting}
\end{Shaded}

\begin{verbatim}
## [1]  TRUE  TRUE  TRUE FALSE
\end{verbatim}

\subsection{Part II: Scoping and related
topics}\label{part-ii-scoping-and-related-topics}

\subsubsection{Scoping I}\label{scoping-i}

The main concept behind this exercise is that of the \emph{Search Path},
which \texttt{R} uses to locate objects when called upon. In order for
\texttt{R} to carry out a command or calculation, it seeks the necessary
information according to a hierarchical path of ``environments''. Each
environment has a ``parent'' environment, to which \texttt{R} moves if
the required information is not yet found. The \texttt{R} workspace is
known as the \emph{Global Environment} and also has a parent, which is
either the most recently loaded package or the base package
(\texttt{package:base}), also known as the ``end'' of the search path.

Below we can observe the importance of the search path with a simple
example:

\begin{Shaded}
\begin{Highlighting}[]
\CommentTok{# We assign numeric values to the vectors x and y in the workspace }
\CommentTok{# which we call the global environment}
\NormalTok{x <-}\StringTok{ }\DecValTok{5}
\NormalTok{y <-}\StringTok{ }\DecValTok{7}

\NormalTok{f <-}\StringTok{ }\NormalTok{function() x *}\StringTok{ }\NormalTok{y}
  \CommentTok{# With no specified argument inputs, the function f follows the search path}
  \CommentTok{# and locates values for x and y in the global environment}
\NormalTok{g <-}\StringTok{ }\NormalTok{function(}\DataTypeTok{x =} \DecValTok{2}\NormalTok{, }\DataTypeTok{y =} \NormalTok{x) x *}\StringTok{ }\NormalTok{y}
  \CommentTok{# A new environment is created within the function g, where arguments for x and y}
  \CommentTok{# are clearly defined}
\end{Highlighting}
\end{Shaded}

Although both functions \texttt{f} and \texttt{g} depend on values for
\texttt{x} and \texttt{y}, they return different results when called:

\begin{Shaded}
\begin{Highlighting}[]
\KeywordTok{f}\NormalTok{()}
\end{Highlighting}
\end{Shaded}

\begin{verbatim}
## [1] 35
\end{verbatim}

\begin{Shaded}
\begin{Highlighting}[]
\KeywordTok{g}\NormalTok{()}
\end{Highlighting}
\end{Shaded}

\begin{verbatim}
## [1] 4
\end{verbatim}

Beginning with function \texttt{f}, if we follow the search path we
begin in the environment within the function itself. Since there is no
information regarding the values of \texttt{x} and \texttt{y},
\texttt{R} moves to the parent environment, which is the global
environment in this case. In the global environment, \texttt{x} takes
the value of \(5\) and \texttt{y} takes the value of \(7\). Thus, the
function returns \(5*7=35\).\\
For function \texttt{g} the search path also begins in the environment
within the function itself. However, in this case there is a defined
value for \texttt{x}, as well as an expression defining a value for
\texttt{y} based on \texttt{x}. The search path ends and the function
returns \(2*2=4\).

By manipulating the arguments of a function, it is also possible to
alter the original search path. We see this when calling the following
function:

\begin{Shaded}
\begin{Highlighting}[]
\KeywordTok{g}\NormalTok{(}\DataTypeTok{y =} \NormalTok{x)}
\end{Highlighting}
\end{Shaded}

\begin{verbatim}
## [1] 10
\end{verbatim}

Looking back at the code for function \texttt{g}, we see the two
arguments \texttt{x} and \texttt{y}. When calling \texttt{g(y\ =\ x)}
however, we are ignoring the first argument, which then defaults to the
value \(2\). When we simply call \texttt{g()}, the \texttt{y\ =\ x}
argument also defaults to a value dependent on \texttt{x}. But by
inputing \texttt{y\ =\ x} manually, we send the search path to the
global environment where \texttt{x} takes the value of \(5\). Thus the
function returns \(2*5=10\).

\subsubsection{Scoping II}\label{scoping-ii}

\subsubsection{Scoping III}\label{scoping-iii}

\subsubsection{Dynamic lookup}\label{dynamic-lookup}

\end{document}
