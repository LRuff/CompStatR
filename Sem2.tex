\documentclass[11,]{article}
\usepackage{lmodern}
\usepackage{amssymb,amsmath}
\usepackage{ifxetex,ifluatex}
\usepackage{fixltx2e} % provides \textsubscript
\ifnum 0\ifxetex 1\fi\ifluatex 1\fi=0 % if pdftex
  \usepackage[T1]{fontenc}
  \usepackage[utf8]{inputenc}
\else % if luatex or xelatex
  \ifxetex
    \usepackage{mathspec}
    \usepackage{xltxtra,xunicode}
  \else
    \usepackage{fontspec}
  \fi
  \defaultfontfeatures{Mapping=tex-text,Scale=MatchLowercase}
  \newcommand{\euro}{€}
\fi
% use upquote if available, for straight quotes in verbatim environments
\IfFileExists{upquote.sty}{\usepackage{upquote}}{}
% use microtype if available
\IfFileExists{microtype.sty}{%
\usepackage{microtype}
\UseMicrotypeSet[protrusion]{basicmath} % disable protrusion for tt fonts
}{}
\usepackage[margin=1in]{geometry}
\usepackage{color}
\usepackage{fancyvrb}
\newcommand{\VerbBar}{|}
\newcommand{\VERB}{\Verb[commandchars=\\\{\}]}
\DefineVerbatimEnvironment{Highlighting}{Verbatim}{commandchars=\\\{\}}
% Add ',fontsize=\small' for more characters per line
\usepackage{framed}
\definecolor{shadecolor}{RGB}{248,248,248}
\newenvironment{Shaded}{\begin{snugshade}}{\end{snugshade}}
\newcommand{\KeywordTok}[1]{\textcolor[rgb]{0.13,0.29,0.53}{\textbf{{#1}}}}
\newcommand{\DataTypeTok}[1]{\textcolor[rgb]{0.13,0.29,0.53}{{#1}}}
\newcommand{\DecValTok}[1]{\textcolor[rgb]{0.00,0.00,0.81}{{#1}}}
\newcommand{\BaseNTok}[1]{\textcolor[rgb]{0.00,0.00,0.81}{{#1}}}
\newcommand{\FloatTok}[1]{\textcolor[rgb]{0.00,0.00,0.81}{{#1}}}
\newcommand{\CharTok}[1]{\textcolor[rgb]{0.31,0.60,0.02}{{#1}}}
\newcommand{\StringTok}[1]{\textcolor[rgb]{0.31,0.60,0.02}{{#1}}}
\newcommand{\CommentTok}[1]{\textcolor[rgb]{0.56,0.35,0.01}{\textit{{#1}}}}
\newcommand{\OtherTok}[1]{\textcolor[rgb]{0.56,0.35,0.01}{{#1}}}
\newcommand{\AlertTok}[1]{\textcolor[rgb]{0.94,0.16,0.16}{{#1}}}
\newcommand{\FunctionTok}[1]{\textcolor[rgb]{0.00,0.00,0.00}{{#1}}}
\newcommand{\RegionMarkerTok}[1]{{#1}}
\newcommand{\ErrorTok}[1]{\textbf{{#1}}}
\newcommand{\NormalTok}[1]{{#1}}
\usepackage{graphicx}
\makeatletter
\def\maxwidth{\ifdim\Gin@nat@width>\linewidth\linewidth\else\Gin@nat@width\fi}
\def\maxheight{\ifdim\Gin@nat@height>\textheight\textheight\else\Gin@nat@height\fi}
\makeatother
% Scale images if necessary, so that they will not overflow the page
% margins by default, and it is still possible to overwrite the defaults
% using explicit options in \includegraphics[width, height, ...]{}
\setkeys{Gin}{width=\maxwidth,height=\maxheight,keepaspectratio}
\ifxetex
  \usepackage[setpagesize=false, % page size defined by xetex
              unicode=false, % unicode breaks when used with xetex
              xetex]{hyperref}
\else
  \usepackage[unicode=true]{hyperref}
\fi
\hypersetup{breaklinks=true,
            bookmarks=true,
            pdfauthor={Group 2: Carlo Michaelis, Patrick Molligo, Lukas Ruff},
            pdftitle={CompStat/R - Paper 2},
            colorlinks=true,
            citecolor=blue,
            urlcolor=blue,
            linkcolor=magenta,
            pdfborder={0 0 0}}
\urlstyle{same}  % don't use monospace font for urls
\setlength{\parindent}{0pt}
\setlength{\parskip}{6pt plus 2pt minus 1pt}
\setlength{\emergencystretch}{3em}  % prevent overfull lines
\setcounter{secnumdepth}{0}

%%% Use protect on footnotes to avoid problems with footnotes in titles
\let\rmarkdownfootnote\footnote%
\def\footnote{\protect\rmarkdownfootnote}

%%% Change title format to be more compact
\usepackage{titling}

% Create subtitle command for use in maketitle
\newcommand{\subtitle}[1]{
  \posttitle{
    \begin{center}\large#1\end{center}
    }
}

\setlength{\droptitle}{-2em}
  \title{CompStat/R - Paper 2}
  \pretitle{\vspace{\droptitle}\centering\huge}
  \posttitle{\par}
  \author{Group 2: Carlo Michaelis, Patrick Molligo, Lukas Ruff}
  \preauthor{\centering\large\emph}
  \postauthor{\par}
  \predate{\centering\large\emph}
  \postdate{\par}
  \date{21 June 2016}



\begin{document}

\maketitle


\subsection{Part I: Functions}\label{part-i-functions}

\subsubsection{Functions I}\label{functions-i}

Below we define a function \texttt{dropNa} which, given an atomic vector
\texttt{x} as an argument, returns \texttt{x} after removing missing
values.

\begin{Shaded}
\begin{Highlighting}[]
\NormalTok{dropNa <-}\StringTok{ }\NormalTok{function(x) \{}
  \CommentTok{# Expects an atomic vector as an argument and returns it without missing}
  \CommentTok{# values}
  \CommentTok{# }
  \CommentTok{# Args:}
  \CommentTok{#   x: atomic vector}
  \CommentTok{#}
  \CommentTok{# Returns:}
  \CommentTok{#   The atomic vector x without missing values}
  
  \CommentTok{# To remove the NAs, we use simple logical subsetting}
  \NormalTok{y <-}\StringTok{ }\NormalTok{x[!}\KeywordTok{is.na}\NormalTok{(x)]}
  
  \CommentTok{# Return y}
  \NormalTok{y}
\NormalTok{\}}
\end{Highlighting}
\end{Shaded}

Let's test our implementation with the following line of code:

\begin{Shaded}
\begin{Highlighting}[]
\KeywordTok{all.equal}\NormalTok{(}\KeywordTok{dropNa}\NormalTok{(}\KeywordTok{c}\NormalTok{(}\DecValTok{1}\NormalTok{, }\DecValTok{2}\NormalTok{, }\DecValTok{3}\NormalTok{, }\OtherTok{NA}\NormalTok{, }\DecValTok{1}\NormalTok{, }\DecValTok{2}\NormalTok{, }\DecValTok{3}\NormalTok{)), }\KeywordTok{c}\NormalTok{(}\DecValTok{1}\NormalTok{, }\DecValTok{2}\NormalTok{, }\DecValTok{3}\NormalTok{, }\DecValTok{1}\NormalTok{, }\DecValTok{2}\NormalTok{, }\DecValTok{3}\NormalTok{))}
\end{Highlighting}
\end{Shaded}

\begin{verbatim}
## [1] TRUE
\end{verbatim}

As we can see from this positive test, our implementation was
successful.

\subsubsection{Functions II}\label{functions-ii}

\paragraph{Part I}\label{part-i}

Below we define a function \texttt{meanVarSdSe} which, given a numeric
vector \texttt{x} as an argument, returns the mean, the variance, the
standard deviation and the standard error of \texttt{x}.

\begin{Shaded}
\begin{Highlighting}[]
\NormalTok{meanVarSdSe <-}\StringTok{ }\NormalTok{function(x) \{}
  \CommentTok{# Expects a numeric vector as an argument and returns the mean,}
  \CommentTok{# the variance, the standard deviation and the standard error}
  \CommentTok{# }
  \CommentTok{# Args:}
  \CommentTok{#   x: numeric vector}
  \CommentTok{#}
  \CommentTok{# Returns:}
  \CommentTok{#   a numerical vector containing mean, variance, standard deviation}
  \CommentTok{#   and standard error of x}
  
  \CommentTok{# We check if x is numeric vector}
  \CommentTok{# If not: stop and throw error}
  \NormalTok{if( !}\KeywordTok{is.numeric}\NormalTok{(x) ) \{}
    \KeywordTok{stop}\NormalTok{(}\StringTok{"Argument needs to be numeric."}\NormalTok{)}
  \NormalTok{\}}
  
  \CommentTok{# Create vector object}
  \NormalTok{y <-}\StringTok{ }\KeywordTok{vector}\NormalTok{()}
  
  \CommentTok{# Calculate mean, variance, standard deviation and standard error}
  \CommentTok{# and save it in y}
  \NormalTok{y[}\DecValTok{1}\NormalTok{] <-}\StringTok{ }\KeywordTok{mean}\NormalTok{(x)}
  \NormalTok{y[}\DecValTok{2}\NormalTok{] <-}\StringTok{ }\KeywordTok{var}\NormalTok{(x)}
  \NormalTok{y[}\DecValTok{3}\NormalTok{] <-}\StringTok{ }\KeywordTok{sd}\NormalTok{(x)}
  \NormalTok{y[}\DecValTok{4}\NormalTok{] <-}\StringTok{ }\NormalTok{y[}\DecValTok{3}\NormalTok{]/}\KeywordTok{sqrt}\NormalTok{(}\KeywordTok{length}\NormalTok{(x))}
  
  \CommentTok{# Set names to vector entries}
  \KeywordTok{names}\NormalTok{(y) <-}\StringTok{ }\KeywordTok{c}\NormalTok{(}\StringTok{"mean"}\NormalTok{, }\StringTok{"var"}\NormalTok{, }\StringTok{"sd"}\NormalTok{, }\StringTok{"se"}\NormalTok{)}
  
  \CommentTok{# Return the numeric vector y}
  \NormalTok{y}
\NormalTok{\}}
\end{Highlighting}
\end{Shaded}

To test the function, we define a numeric vector, which contains numbers
from \(1\) to \(100\), and use it as an argument for our function
\texttt{meanVarSdSe}:

\begin{Shaded}
\begin{Highlighting}[]
\NormalTok{x <-}\StringTok{ }\DecValTok{1}\NormalTok{:}\DecValTok{100}
\KeywordTok{meanVarSdSe}\NormalTok{(x)}
\end{Highlighting}
\end{Shaded}

\begin{verbatim}
##       mean        var         sd         se 
##  50.500000 841.666667  29.011492   2.901149
\end{verbatim}

Finally we can confirm that the result is of class \texttt{numeric}:

\begin{Shaded}
\begin{Highlighting}[]
\KeywordTok{class}\NormalTok{(}\KeywordTok{meanVarSdSe}\NormalTok{(x))}
\end{Highlighting}
\end{Shaded}

\begin{verbatim}
## [1] "numeric"
\end{verbatim}

\paragraph{Part II}\label{part-ii}

Now we will have a look at the case below. We would expect that the
function will return a vector with \texttt{NA}s:

\begin{Shaded}
\begin{Highlighting}[]
\NormalTok{x <-}\StringTok{ }\KeywordTok{c}\NormalTok{(}\OtherTok{NA}\NormalTok{, }\DecValTok{1}\NormalTok{:}\DecValTok{100}\NormalTok{)}
\KeywordTok{meanVarSdSe}\NormalTok{(x)}
\end{Highlighting}
\end{Shaded}

\begin{verbatim}
## mean  var   sd   se 
##   NA   NA   NA   NA
\end{verbatim}

The reason for the result is that the functions \texttt{mean()},
\texttt{var()} and \texttt{sd()} use \texttt{na.rm = FALSE} as default,
which means that missing values are not removed. If the vector
\texttt{x} contains a missing value, the \texttt{mean()} function (as
well as \texttt{var()} and \texttt{sd()}) will just return \texttt{NA}
to inform about missing values. In the case of calculating standard
error we use the result from our \texttt{sd()} function and calculate an
\texttt{NA} value with some other numeric values, which will ultimately
result in \texttt{NA} again.

To solve the problem, we can add \texttt{na.rm = TRUE} to these three
functions. To make this optional, we will improve the
\texttt{meanVarSdSe} function from above as follows:

\begin{Shaded}
\begin{Highlighting}[]
\NormalTok{meanVarSdSe <-}\StringTok{ }\NormalTok{function(x, ...) \{}
  \CommentTok{# Expects a numeric vector and flag to handle missing values as an argument}
  \CommentTok{# and returns the mean, the variance, the standard deviation}
  \CommentTok{# and the standard error}
  \CommentTok{# }
  \CommentTok{# Args:}
  \CommentTok{#   x: numeric vector, na.rm: boolean}
  \CommentTok{#}
  \CommentTok{# Returns:}
  \CommentTok{#   a numerical vector containing mean, variance, standard deviation}
  \CommentTok{#   and standard error of x}
  
  \CommentTok{# We check if x is numeric vector}
  \CommentTok{# If not: stop and throw error}
  \NormalTok{if( !}\KeywordTok{is.numeric}\NormalTok{(x) ) \{}
    \KeywordTok{stop}\NormalTok{(}\StringTok{"Argument needs to be numeric."}\NormalTok{)}
  \NormalTok{\}}
  
  \CommentTok{# Create vector object}
  \NormalTok{y <-}\StringTok{ }\KeywordTok{vector}\NormalTok{()}
  
  \CommentTok{# Calculate mean, variance, standard deviation and standard error}
  \CommentTok{# and save it in y}
  \NormalTok{y[}\DecValTok{1}\NormalTok{] <-}\StringTok{ }\KeywordTok{mean}\NormalTok{(x, ...)}
  \NormalTok{y[}\DecValTok{2}\NormalTok{] <-}\StringTok{ }\KeywordTok{var}\NormalTok{(x, ...)}
  \NormalTok{y[}\DecValTok{3}\NormalTok{] <-}\StringTok{ }\KeywordTok{sd}\NormalTok{(x, ...)}
  \NormalTok{y[}\DecValTok{4}\NormalTok{] <-}\StringTok{ }\NormalTok{y[}\DecValTok{3}\NormalTok{]/}\KeywordTok{sqrt}\NormalTok{(}\KeywordTok{length}\NormalTok{(x) -}\StringTok{ }\KeywordTok{sum}\NormalTok{(}\KeywordTok{is.na}\NormalTok{(x)))}
  
  \CommentTok{# Set names to vector entries}
  \KeywordTok{names}\NormalTok{(y) <-}\StringTok{ }\KeywordTok{c}\NormalTok{(}\StringTok{"mean"}\NormalTok{, }\StringTok{"var"}\NormalTok{, }\StringTok{"sd"}\NormalTok{, }\StringTok{"se"}\NormalTok{)}
  
  \CommentTok{# Return the numeric vector y}
  \NormalTok{y}
\NormalTok{\}}
\end{Highlighting}
\end{Shaded}

We define the function with an ellipse \texttt{...}. Our function can
now receive multiple arguments after the first input \texttt{x}. These
arguments are used in \texttt{mean()}, \texttt{var()} and \texttt{sd()}.
If we want to remove missing values in all of these functions (to get a
result in the case of missing values), we can pass \texttt{na.rm = TRUE}
as another argument, such as here:
\texttt{meanVarSdSe(x, na.rm = TRUE)}. We just have to be aware of
\texttt{length(x)} in this case. If we want to have the same result as
above we have to remove the sum of \texttt{NA} values from the length of
\texttt{x}. Otherwise the function will calculate a different result
than in Part I, because then lentgh differs.

Let's confirm the result:

\begin{Shaded}
\begin{Highlighting}[]
\KeywordTok{meanVarSdSe}\NormalTok{(}\KeywordTok{c}\NormalTok{(x, }\OtherTok{NA}\NormalTok{), }\DataTypeTok{na.rm =} \OtherTok{TRUE}\NormalTok{)}
\end{Highlighting}
\end{Shaded}

\begin{verbatim}
##       mean        var         sd         se 
##  50.500000 841.666667  29.011492   2.901149
\end{verbatim}

\paragraph{Part III}\label{part-iii}

Now we will use the function \texttt{dropNa} from Functions I to deal
with missing values in \texttt{meanVarSdSe}.

\begin{Shaded}
\begin{Highlighting}[]
\NormalTok{meanVarSdSe <-}\StringTok{ }\NormalTok{function(x) \{}
  \CommentTok{# Expects a numeric vector as an argument and returns the mean,}
  \CommentTok{# the variance, the standard deviation and the standard error}
  \CommentTok{# it also removes missing values if x contains some}
  \CommentTok{# }
  \CommentTok{# Args:}
  \CommentTok{#   x: numeric vector}
  \CommentTok{#}
  \CommentTok{# Returns:}
  \CommentTok{#   a numerical vector containing mean, variance, standard deviation}
  \CommentTok{#   and standard error of x}
  
  \CommentTok{# We check if x is numeric vector}
  \CommentTok{# If not: stop and throw error}
  \NormalTok{if( !}\KeywordTok{is.numeric}\NormalTok{(x) ) \{}
    \KeywordTok{stop}\NormalTok{(}\StringTok{"Argument needs to be numeric."}\NormalTok{)}
  \NormalTok{\}}
  
  \CommentTok{# We check if x contains missing values}
  \CommentTok{# If so: remove missing values using dropNA}
  \NormalTok{if( }\KeywordTok{sum}\NormalTok{(}\KeywordTok{is.na}\NormalTok{(x)) >}\StringTok{ }\DecValTok{0} \NormalTok{) \{}
    \NormalTok{x <-}\StringTok{ }\KeywordTok{dropNa}\NormalTok{(x)}
  \NormalTok{\}}
  
  \CommentTok{# Create vector object}
  \NormalTok{y <-}\StringTok{ }\KeywordTok{vector}\NormalTok{()}
  
  \CommentTok{# Calculate mean, variance, standard deviation and standard error}
  \CommentTok{# and save it in y}
  \NormalTok{y[}\DecValTok{1}\NormalTok{] <-}\StringTok{ }\KeywordTok{mean}\NormalTok{(x)}
  \NormalTok{y[}\DecValTok{2}\NormalTok{] <-}\StringTok{ }\KeywordTok{var}\NormalTok{(x)}
  \NormalTok{y[}\DecValTok{3}\NormalTok{] <-}\StringTok{ }\KeywordTok{sd}\NormalTok{(x)}
  \NormalTok{y[}\DecValTok{4}\NormalTok{] <-}\StringTok{ }\NormalTok{y[}\DecValTok{3}\NormalTok{]/}\KeywordTok{sqrt}\NormalTok{(}\KeywordTok{length}\NormalTok{(x))}
  
  \CommentTok{# Set names to vector entries}
  \KeywordTok{names}\NormalTok{(y) <-}\StringTok{ }\KeywordTok{c}\NormalTok{(}\StringTok{"mean"}\NormalTok{, }\StringTok{"var"}\NormalTok{, }\StringTok{"sd"}\NormalTok{, }\StringTok{"se"}\NormalTok{)}
  
  \CommentTok{# Return the numeric vector y}
  \NormalTok{y}
\NormalTok{\}}
\end{Highlighting}
\end{Shaded}

We used the function from Part I and added a condition which checks if
we have missing values in \texttt{x}, using \texttt{is.na}. If the sum
of \texttt{NA} values is greater than \(0\) (i.e.,if there is one or
more missing value), we use the function \texttt{dropNA} from the first
exercise to remove all missing values. The remaining code of the
function can remain as above in Part I.

We can confirm the result:

\begin{Shaded}
\begin{Highlighting}[]
\KeywordTok{meanVarSdSe}\NormalTok{(}\KeywordTok{c}\NormalTok{(x, }\OtherTok{NA}\NormalTok{))}
\end{Highlighting}
\end{Shaded}

\begin{verbatim}
##       mean        var         sd         se 
##  50.500000 841.666667  29.011492   2.901149
\end{verbatim}

\subsubsection{Functions III}\label{functions-iii}

In this section we define an infix function \texttt{\%or\%}. This
function should behave like the logical operator \texttt{\textbar{}}.

\begin{Shaded}
\begin{Highlighting}[]
\CommentTok{# Define infix function %or%}
\StringTok{`}\DataTypeTok\StringTok{`} \NormalTok{<-}\StringTok{ }\NormalTok{function(a, b) \{}
  \CommentTok{# Check if vector a and b is logical}
  \NormalTok{if( !(}\KeywordTok{is.logical}\NormalTok{(a) &}\StringTok{ }\KeywordTok{is.logical}\NormalTok{(b)) ) \{}
    \KeywordTok{stop}\NormalTok{(}\StringTok{"a and/or b have to be logical vectors."}\NormalTok{)}
  \NormalTok{\}}
  
  \CommentTok{# Use ifelse to calculate result and return it directly}
  \CommentTok{# If the sum of the entry of vector a and the entry of vector b}
  \CommentTok{# is greater than or equal to 1, set result to TRUE, otherwise to FALSE}
  \KeywordTok{ifelse}\NormalTok{(a +}\StringTok{ }\NormalTok{b >=}\StringTok{ }\DecValTok{1}\NormalTok{,  }\OtherTok{TRUE}\NormalTok{, }\OtherTok{FALSE}\NormalTok{)}
\NormalTok{\}}
\end{Highlighting}
\end{Shaded}

First we check if we have logical vectors. If \texttt{a} and/or
\texttt{b} are not logical, we leave the function and throw an error.
Otherwise we can calculate the \texttt{or} operation using the
\texttt{ifelse} function and return the result directly after
calculation. Inside of the \texttt{ifelse} function, the first argument
checks the condition if the sum of the values \texttt{a} and \texttt{b}
are greater than or equal to \(1\), where \texttt{TRUE} is equal to
\(1\) and \texttt{FALSE} is equal to \(0\).

To confirm the function, we test an example:

\begin{Shaded}
\begin{Highlighting}[]
\KeywordTok{c}\NormalTok{(}\OtherTok{TRUE}\NormalTok{, }\OtherTok{FALSE}\NormalTok{, }\OtherTok{TRUE}\NormalTok{, }\OtherTok{FALSE}\NormalTok{) %or%}\StringTok{ }\KeywordTok{c}\NormalTok{(}\OtherTok{TRUE}\NormalTok{, }\OtherTok{TRUE}\NormalTok{, }\OtherTok{FALSE}\NormalTok{, }\OtherTok{FALSE}\NormalTok{)}
\end{Highlighting}
\end{Shaded}

\begin{verbatim}
## [1]  TRUE  TRUE  TRUE FALSE
\end{verbatim}

\subsection{Part II: Scoping and related
topics}\label{part-ii-scoping-and-related-topics}

\subsubsection{Scoping I}\label{scoping-i}

The main concept behind this exercise is the \emph{Search Path}, which
\texttt{R} uses to locate objects when called upon. In order for
\texttt{R} to carry out a command or calculation, it seeks the necessary
information according to a hierarchical path of \emph{environments}.
Each environment has a \emph{parent}, to which \texttt{R} moves if the
required information is not yet found. The \texttt{R} workspace is known
as the \emph{Global Environment} and also has a parent, which is the
most recently loaded package. If there are no longer any loaded
packages, then the search path \emph{ends} at the final parent
environment, the base package (\texttt{package:base}) which just has the
empty environment as parent.

Below we can observe the importance of the search path with a simple
example:

\begin{Shaded}
\begin{Highlighting}[]
\CommentTok{# Assign numeric values to the vectors x and y in the workspace }
\CommentTok{# which we call the global environment}
\NormalTok{x <-}\StringTok{ }\DecValTok{5}
\NormalTok{y <-}\StringTok{ }\DecValTok{7}

\NormalTok{f <-}\StringTok{ }\NormalTok{function() x *}\StringTok{ }\NormalTok{y}
  \CommentTok{# With no specified argument inputs, the function f follows the search path}
  \CommentTok{# and locates values for x and y in the global environment}
\NormalTok{g <-}\StringTok{ }\NormalTok{function(}\DataTypeTok{x =} \DecValTok{2}\NormalTok{, }\DataTypeTok{y =} \NormalTok{x) x *}\StringTok{ }\NormalTok{y}
  \CommentTok{# A new environment is created within the function g, where arguments for x and y}
  \CommentTok{# are clearly defined}
\end{Highlighting}
\end{Shaded}

Although both functions \texttt{f} and \texttt{g} depend on values for
\texttt{x} and \texttt{y}, they return different results when called:

\begin{Shaded}
\begin{Highlighting}[]
\CommentTok{# call 1}
\KeywordTok{f}\NormalTok{()}
\end{Highlighting}
\end{Shaded}

\begin{verbatim}
## [1] 35
\end{verbatim}

\begin{Shaded}
\begin{Highlighting}[]
\CommentTok{# call 2}
\KeywordTok{g}\NormalTok{()}
\end{Highlighting}
\end{Shaded}

\begin{verbatim}
## [1] 4
\end{verbatim}

Beginning with function \texttt{f}, if we follow the search path we
begin in the temporal local environment within the function itself.
Since there is no information regarding the values of \texttt{x} and
\texttt{y}, \texttt{R} moves to the parent environment, which is the
global environment in this case. In the global environment, \texttt{x}
takes the value of \(5\) and \texttt{y} takes the value of \(7\). Thus,
the function returns \(5 \cdot 7=35\).

For function \texttt{g} the search path also begins in the local
environment within the function itself. However, in this case there is a
defined value for \texttt{x}, as well as an expression defining a value
for \texttt{y} based on \texttt{x}. The search path ends and the
function returns \(2 \cdot 2=4\).

By manipulating the arguments of a function, it is also possible to
alter the original search path. We see this when calling the following
function:

\begin{Shaded}
\begin{Highlighting}[]
\CommentTok{# call 3}
\KeywordTok{g}\NormalTok{(}\DataTypeTok{y =} \NormalTok{x)}
\end{Highlighting}
\end{Shaded}

\begin{verbatim}
## [1] 10
\end{verbatim}

Looking back at the code for function \texttt{g}, we see the two
arguments \texttt{x} and \texttt{y}. When calling \texttt{g(y = x)}
however, we are omitting the first argument \texttt{x}, which then
defaults to the value \(2\), defined in the local environment of the
function.\\When we simply call \texttt{g()}, the \texttt{y = x} argument
also defaults to a local value dependent on local \texttt{x}. But by
inputing the argument \texttt{y = x} manually while calling, we send the
search path to the global environment where \texttt{x} takes the global
value of \(5\). Thus the function returns \(2 \cdot 5=10\).

\subsubsection{Scoping II}\label{scoping-ii}

In this exercise we once again see the importance of understanding the
search path and how \texttt{R} carries out tasks according to the
environment in which it is working. Especially important is the
\emph{naming} of objects and functions. As discussed in the previous
section, the ultimate parent environment to use is
\texttt{package:base}, which contains the commonly used and most
fundamental functions in \texttt{R}. Since the global environment
(workspace) is separate from \texttt{package:base}, it is possible to
name new objects in the workspace using previously defined functions
from the base. As long as there is no overlap \emph{within} an
enviornment, nothing will be overwritten, it will just be masked. In the
following example we see again why the search path is so important when
defining objects:

\begin{Shaded}
\begin{Highlighting}[]
\CommentTok{# Define matrix t, where the number of columns is selected as 3}
\CommentTok{# and the matrix is filled row-wise}
\NormalTok{t <-}\StringTok{ }\KeywordTok{matrix}\NormalTok{(}\DecValTok{1}\NormalTok{:}\DecValTok{6}\NormalTok{, }\DataTypeTok{ncol =} \DecValTok{3}\NormalTok{, }\DataTypeTok{byrow =} \OtherTok{TRUE}\NormalTok{)}

\CommentTok{# Print matrix t}
\NormalTok{t}
\end{Highlighting}
\end{Shaded}

\begin{verbatim}
##      [,1] [,2] [,3]
## [1,]    1    2    3
## [2,]    4    5    6
\end{verbatim}

As expected, printing \texttt{t} returns a \(2\times3\) matrix filled by
row using the numbers one through six. Let's see what happens if we
treat \texttt{t} like a function:

\begin{Shaded}
\begin{Highlighting}[]
\CommentTok{# Print t(t), which should transpose matrix t}
\KeywordTok{t}\NormalTok{(t)}
\end{Highlighting}
\end{Shaded}

\begin{verbatim}
##      [,1] [,2]
## [1,]    1    4
## [2,]    2    5
## [3,]    3    6
\end{verbatim}

The result is a \(3\times2\) matrix filled by column using the numbers
one through six. In other words, we have printed the transpose of the
original matrix \texttt{t}, which we had defined in the global
environment. Since \texttt{t} is a defined matrix and not a function,
\texttt{R} will ignore the \texttt{t} in the global environment while
searching for function \texttt{t}. \texttt{R} follows the search path
from the global environment to the earlier parents and finds function
\texttt{t} in \texttt{package:base}. In the base environment, the
function \texttt{t()} returns the transpose of the given matrix.

\subsubsection{Scoping III}\label{scoping-iii}

In the previous exercises we observed how \texttt{R} searches through a
chain of environments to locate objects and information. In this next
exercise, we investigate what happens when different objects are defined
identically within the \emph{same} environment. Here we are defining
objects in the global environment (workspace):

\begin{Shaded}
\begin{Highlighting}[]
\CommentTok{# Define a function t in the global environment}
\NormalTok{t <-}\StringTok{ }\NormalTok{function(...) }\KeywordTok{matrix}\NormalTok{(...)}

\CommentTok{# Define a matrix T using the above t function}
\CommentTok{# with the desired input arguments}
\NormalTok{T <-}\StringTok{ }\KeywordTok{t}\NormalTok{(}\DecValTok{1}\NormalTok{:}\DecValTok{6}\NormalTok{, }\DataTypeTok{ncol =} \DecValTok{3}\NormalTok{, }\DataTypeTok{byrow =} \OtherTok{TRUE}\NormalTok{)}

\CommentTok{# Print result of T}
\NormalTok{T}
\end{Highlighting}
\end{Shaded}

\begin{verbatim}
##      [,1] [,2] [,3]
## [1,]    1    2    3
## [2,]    4    5    6
\end{verbatim}

As expected, printing \texttt{T} returns a \(2 \times 3\) matrix filled
by row using the numbers one through six. Now let's enter \texttt{T}
into the function \texttt{t}:

\begin{Shaded}
\begin{Highlighting}[]
\CommentTok{# Call defined function t with argument T}
\KeywordTok{t}\NormalTok{(T)}
\end{Highlighting}
\end{Shaded}

\begin{verbatim}
##      [,1]
## [1,]    1
## [2,]    4
## [3,]    2
## [4,]    5
## [5,]    3
## [6,]    6
\end{verbatim}

Since \texttt{t} is a function we have defined in our workspace (global
environment), \texttt{t()} takes \texttt{T} as an argument input and
returns a column vector containing the numbers one through six (note:
\texttt{t} and \texttt{T} are different objects, because \texttt{R} is
case sensitive). The transpose function \texttt{t()} from the base
environment is now masked by our own function, saved in the global
environment and now is just an alias of \texttt{matrix()} (as we defined
it). Since the \texttt{matrix()} function has the default value
\texttt{ncol = 1} and we were not giving any argument, it creates a
matrix out of the data from \texttt{T} (values \(1\) to \(6\)) and put
it in just one column.

Let's now see what would happen if we had defined \texttt{T} instead as
\texttt{t}:

\begin{Shaded}
\begin{Highlighting}[]
\CommentTok{# Define a function t in the global environment}
\NormalTok{t <-}\StringTok{ }\NormalTok{function(...) }\KeywordTok{matrix}\NormalTok{(...)}

\CommentTok{# We now define t as the following matrix using the t function from above}
\NormalTok{t <-}\StringTok{ }\KeywordTok{t}\NormalTok{(}\DecValTok{1}\NormalTok{:}\DecValTok{6}\NormalTok{, }\DataTypeTok{ncol =} \DecValTok{3}\NormalTok{, }\DataTypeTok{byrow =} \OtherTok{TRUE}\NormalTok{)}
\CommentTok{# Although we used the function t to define the new matrix t}
\CommentTok{# both are defined in the global environment}

\CommentTok{# Call defined function t with argument t}
\KeywordTok{t}\NormalTok{(t)}
\end{Highlighting}
\end{Shaded}

\begin{verbatim}
##      [,1] [,2]
## [1,]    1    4
## [2,]    2    5
## [3,]    3    6
\end{verbatim}

Since two objects (independent of their type) cannot have the same name
within our global environment, the new matrix \texttt{t} overwrites the
original function. In our global environment, \texttt{t} is now a
defined matrix and no longer a function, it was replaced. The search
path now moves down (is looking earlier in path), until it finds
\texttt{t}, defined as the transpose function (seen earlier), in base
environment. It is therefore clear why we receive the same result as in
the previous exercise when printing \texttt{t(t)} here.

This entire concept can be referred to as \emph{name masking}. We can
think of the transpose function \texttt{t} in the base environment as
the \emph{original} function. Each time a new object \texttt{t} is
created in later environments, the original is \emph{masked}, but not
overwritten. So if the search path is led back to the base environment,
the original function can still be located.

\subsubsection{Dynamic lookup}\label{dynamic-lookup}

\texttt{R} searches for objects while it runs the code, which is called
\emph{dynamic lookup}. A well defined function should only process local
variables. In other words it should only depend on the given arguments.
This property is called \emph{self containment}. If we use not well
defined functions, it can lead to inconsistent results or we risk
errors. For illustration, we can use the example:

\begin{Shaded}
\begin{Highlighting}[]
\CommentTok{# first we remove everything from workspace}
\CommentTok{# to avoid conflicts with the stuff above}
\KeywordTok{rm}\NormalTok{(}\DataTypeTok{list =} \KeywordTok{ls}\NormalTok{(}\DataTypeTok{all.names=}\OtherTok{TRUE}\NormalTok{))}

\CommentTok{# we define a function with two arguments}
\CommentTok{# which is well defined, it just processes local variables}
\NormalTok{f <-}\StringTok{ }\NormalTok{function(x, }\DataTypeTok{y =} \NormalTok{x +}\StringTok{ }\DecValTok{1}\NormalTok{) x +}\StringTok{ }\NormalTok{y}

\CommentTok{# we set variable x to 3 and call the function where the first argument is 2}
\CommentTok{# to see that we just use the local variable}
\NormalTok{x <-}\StringTok{ }\DecValTok{3}
\KeywordTok{f}\NormalTok{(}\DecValTok{2}\NormalTok{)  }\CommentTok{# call 1}
\end{Highlighting}
\end{Shaded}

\begin{verbatim}
## [1] 5
\end{verbatim}

\begin{Shaded}
\begin{Highlighting}[]
\CommentTok{# we set variable x to 5 and call the function where the first argument is 2}
\CommentTok{# to see that we just use the local variable again}
\NormalTok{x <-}\StringTok{ }\DecValTok{5}
\KeywordTok{f}\NormalTok{(}\DecValTok{2}\NormalTok{)  }\CommentTok{# call 2}
\end{Highlighting}
\end{Shaded}

\begin{verbatim}
## [1] 5
\end{verbatim}

As mentioned in the comments, we first define a well defined function
\texttt{f()}. The function has two arguments, where \texttt{x} has no
default and \texttt{y} has a default which depends on the local
\texttt{x}, where \texttt{y = x + 1}. If we set a global variable
\texttt{x} (in global environment), it will not be used in the function,
except we pass it as an argument, which is not the case. In both calls,
\emph{call 1} and \emph{call 2}, we pass \(2\) as an argument, which
leads to the same result in both cases. The global variables \texttt{x}
are not used at all.

\begin{Shaded}
\begin{Highlighting}[]
\CommentTok{# we define a function with one argument}
\CommentTok{# which is NOT well defined, variable x is not local}
\NormalTok{f <-}\StringTok{ }\NormalTok{function(}\DataTypeTok{y =} \NormalTok{x +}\StringTok{ }\DecValTok{1}\NormalTok{) x +}\StringTok{ }\NormalTok{y}

\CommentTok{# we set variable x to 3 and overwrite the default argument with 2}
\CommentTok{# to see that we use the global variable}
\NormalTok{x <-}\StringTok{ }\DecValTok{3}
\KeywordTok{f}\NormalTok{(}\DecValTok{2}\NormalTok{)  }\CommentTok{# call 3}
\end{Highlighting}
\end{Shaded}

\begin{verbatim}
## [1] 5
\end{verbatim}

\begin{Shaded}
\begin{Highlighting}[]
\CommentTok{# we set variable x to 5 and overwrite the default argument with 2}
\CommentTok{# to see that we use the global variable again}
\NormalTok{x <-}\StringTok{ }\DecValTok{5}
\KeywordTok{f}\NormalTok{(}\DecValTok{2}\NormalTok{)  }\CommentTok{# call 4}
\end{Highlighting}
\end{Shaded}

\begin{verbatim}
## [1] 7
\end{verbatim}

\begin{Shaded}
\begin{Highlighting}[]
\CommentTok{# variable x is still set to 5 and we call the function with default argument}
\CommentTok{# to see that we use the global variable again}
\KeywordTok{f}\NormalTok{()  }\CommentTok{# call 5}
\end{Highlighting}
\end{Shaded}

\begin{verbatim}
## [1] 11
\end{verbatim}

As mentioned in the comments, we first define a function \texttt{f()}
which is \emph{not} well defined. The function has one argument
\texttt{y} which has a default that depends on the global \texttt{x},
where the formula is the same then before \texttt{y = x + 1}. In this
case we \emph{have} to set a global variable \texttt{x}, otherwise we
will run into errors, like
\texttt{Error in f(2) : object \textquotesingle{}x\textquotesingle{} not found}.

In \emph{call 3} we set the argument \texttt{y} to \(2\). Inside of the
function the pre-defined global variable \texttt{x} is used, which is
set to \(3\). The function calculates \texttt{x + y} which is
\(3 + 2 = 5\) in this case.

In \emph{call 4} we set the argument \texttt{y} to \(2\) again. Inside
of the function the pre-defined global variable \texttt{x} is used
again, which is set to \(5\) in this case. The function calculates
\texttt{x + y} which is \(5 + 2 = 7\) in this case.

In \emph{call 5} we use the default argument \texttt{y = x + 1}, which
depends on global \texttt{x}. Inside of the function the formula changes
to \texttt{x + x + 1}, which depends on the pre-defined global variable
\texttt{x} again, which is still set to \(5\). The calculation
\(5 + 5 + 1\) results in \(11\).

We can state that well defined functions are so nice, that we should
always use them.

\end{document}
